\documentclass{article}
\usepackage{graphicx} % Required for inserting images

% setting lainnya
\usepackage{mystyles}

\title{Pembahasan Soal Tutorial PDP Bab 2 \\
Tahun Ajaran 2024/2025 Semester Genap}
\author{Asisten dosen: Citius Vienny}

\begin{document}

\maketitle

\noindent
%Semua berkas atau \textit{file} responsi bisa diakses melalui folder Google Drive berikut: \\
%\href{https://bit.ly/RespoAljabar2024Ganjil}{Responsi Aljabar TA 2024/2025 Ganjil} \\
%\url{https://bit.ly/RespoAljabar2024Ganjil} \\

\noindent
\textbf{Soal bahasan:}
\begin{enumerate}  
\item Tunjukkanlah bahwa \( u(x,y) = \ln \sqrt{x^2 + y^2} \) memenuhi bentuk umum persamaan Laplace untuk semua \((x,y) \neq (0,0)\).  

\item Tentukanlah fungsi \( u = u(x,t) \) yang memenuhi  
\begin{align*}  
u_{xx} &= 0, \quad 0 < x < 1, \, t > 0, \\  
u(0,t) &= t^2, \quad u(1,t) = 1, \, t > 0.  
\end{align*}  

\item Tunjukkan bahwa persamaan \( u_t = u_{xx} + u_x \) dapat direduksi menjadi persamaan panas dengan menggunakan permisalan \( w = e^{u} \).  

\item Tentukanlah solusi dari  
\[  
2u_{xx} - 4u_{xt} + u_x = 0,  
\]  
dalam bentuk dua fungsi sebarang.  

\item Klasifikasikan Persamaan Differensial Parsial berikut:  
\[  
u_{xx} - 6u_{xy} + 12u_{yy} = 0,  
\]  
lalu tentukanlah permisalan yang tepat untuk merubah persamaan di atas menjadi persamaan Laplace.  

\item Klasifikasikanlah Persamaan Differensial Parsial (PDP) berikut:  
\[  
u_{xx} + 2ku_{xt} + k^2 u_{tt} = 0, \quad k \neq 0,  
\]  
kemudian tentukanlah transformasi \(\xi = x + bt\), \(\tau = x + dt\) dari variabel independent yang dapat mentransformasikan persamaan tersebut menjadi persamaan sederhana dalam bentuk \( U_{\xi\xi} = 0 \). Tentukan juga solusi untuk persamaan yang diberikan dalam bentuk dua fungsi sebarang.  

\item Selesaikanlah masalah berikut:  
\begin{enumerate}  
\item  
\begin{align*}  
u_t + xu_x &= -tu, \quad t > 0, \\  
u(x,0) &= f(x), \quad x \in \mathbb{R}.  
\end{align*}  
\item  
\begin{align*}  
u_t + u_x &= -tu, \quad t > 0, \\  
u(x,0) &= f(x), \quad x \in \mathbb{R}.  
\end{align*}  
\end{enumerate}  

\item Tentukanlah solusi umum dari persamaan:  
\[  
u_t + cu_x = f(x)u.  
\]  
Solusi Anda boleh mengandung integral.  

\item Misalkan \( u = u(x,t) \) memenuhi:  
\begin{align*}  
u_t &= ku_{xx}, \quad 0 < x < l, \, t > 0, \\  
u(0,t) &= u(l,t) = 0, \quad t > 0, \\  
u(x,0) &= u_0(x), \quad 0 \leq x \leq l.  
\end{align*}  
Tunjukkan bahwa  
\[  
\int_0^l u(x,t)^2 \, dx \leq \int_0^l u_0(x)^2 \, dx, \quad t > 0.  
\]  

\item Persamaan panas pada suatu batang logam dengan sumber panas sebesar satu satuan diberikan oleh:  
\begin{align*}  
u_t &= ku_{xx} + 1, \quad 0 < x < 1, \, t > 0, \\  
u(0,t) &= 1, \quad u(1,t) = 1, \quad t > 0.  
\end{align*}  
Tentukanlah temperaturnya saat kondisi setimbang. Apakah nilai awalnya berpengaruh?  
\end{enumerate}
\end{document}